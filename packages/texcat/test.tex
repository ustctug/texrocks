\documentclass{article}
\usepackage{fancyvrb}
\usepackage{color}
\usepackage{amsfonts}
\directlua{
  local outputs = require'texcat'.main{'test.lua', 'test.tex', '--math-escape'}
  token.set_macro('mylua', outputs[1])
  token.set_macro('mytex', outputs[2])
}
\begin{document}

\TeX{}Cat is a rewrite of pygments. For a test file:

\VerbatimInput[
  numbers=left,
  commandchars=\\\{\},
  frame=single,
  label=test.lua,
  codes={\catcode`\$=3\catcode`\^=7\catcode`\_=8\relax},
]{\mylua}

We can display it in \LaTeX{} document:

\VerbatimInput[
  numbers=left,
  commandchars=\\\{\},
  frame=lines,
  label=test.tex,
]{\mytex}

\end{document}
